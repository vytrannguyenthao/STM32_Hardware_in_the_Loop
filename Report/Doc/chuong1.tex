\chapter{GIỚI THIỆU ĐỀ TÀI}
\section{Tổng quan về đề tài}
Trong quá trình phát triển hệ thống nhúng, việc kiểm thử và xác minh chức năng của phần mềm và phần cứng là một yêu cầu bắt buộc nhằm đảm bảo tính đúng đắn và độ tin cậy của hệ thống. Tuy nhiên, ở các giai đoạn đầu của quá trình phát triển, phần cứng thực tế thường chưa hoàn thiện hoặc chưa sẵn sàng để triển khai kiểm thử toàn diện. Việc phụ thuộc quá nhiều vào phần cứng trong giai đoạn này không chỉ làm chậm tiến độ phát triển mà còn làm tăng chi phí và hạn chế khả năng lặp lại các kịch bản kiểm thử.

Bên cạnh đó, các hệ thống nhúng hiện đại ngày càng trở nên phức tạp với sự tích hợp của nhiều ngoại vi và giao tiếp khác nhau. Việc kiểm thử trực tiếp các chức năng liên quan đến những ngoại vi này trên phần cứng thực tế gặp nhiều khó khăn, đặc biệt đối với các kịch bản lỗi hiếm gặp, các điều kiện biên hoặc các tình huống không an toàn. Trong nhiều trường hợp, việc tái hiện chính xác các điều kiện vận hành này trên hệ thống thực là không khả thi hoặc tiềm ẩn rủi ro cho thiết bị.

Xuất phát từ những hạn chế trên, các phương pháp kiểm thử dựa trên mô phỏng đã được nghiên cứu và phát triển nhằm hỗ trợ quá trình thiết kế và triển khai hệ thống nhúng. Trong đó, phương pháp mô phỏng phần cứng trong vòng lặp - HIL (Hardware in the Loop), được xem là một giải pháp hiệu quả, cho phép kết hợp giữa phần cứng thực và môi trường mô phỏng trong cùng một hệ thống kiểm thử.

Hệ thống HIL hoạt động bằng cách kết nối trực tiếp phần cứng thực của hệ thống nhúng, thường được gọi là thiết bị cần kiểm thử, với một môi trường mô phỏng có nhiệm vụ tái hiện hành vi của các thành phần bên ngoài DUT. Môi trường mô phỏng này có thể tạo ra các tín hiệu đầu vào, xử lý phản hồi đầu ra và giao tiếp với DUT thông qua các giao diện phần cứng, từ đó tạo nên một vòng lặp khép kín giữa phần cứng và mô hình mô phỏng.

Thông qua HIL, các chức năng của hệ thống nhúng có thể được kiểm thử một cách toàn diện trong điều kiện gần với thực tế mà không cần triển khai đầy đủ hệ thống vật lý. Phương pháp này cho phép phát hiện sớm các lỗi về logic, lỗi giao tiếp và các sai lệch trong thiết kế, đồng thời hỗ trợ đánh giá hiệu năng và độ ổn định của hệ thống. Nhờ đó, quá trình phát triển hệ thống nhúng được tối ưu hóa về mặt thời gian, chi phí và độ an toàn trước khi đưa vào ứng dụng thực tế.

\section{Tình hình nghiên cứu trong và ngoài nước}
Trong những năm gần đây, mô phỏng phần cứng trong vòng lặp đã được nghiên cứu và ứng dụng ngày càng nhiều trong lĩnh vực thiết bị nhúng và IoT, nhằm phục vụ kiểm thử firmware và giao tiếp ngoại vi. Thay vì tập trung vào các hệ thống quy mô lớn như ô tô hay hàng không, nhiều nghiên cứu hướng đến HIL chi phí thấp, phù hợp với vi điều khiển và hệ thống nhúng phổ biến.

Trên thế giới, nhiều nền tảng HIL chuyên dụng đã được phát triển để phục vụ nghiên cứu và kiểm thử hệ thống nhúng. Các thiết bị này cho phép mô phỏng môi trường vận hành và các thành phần bên ngoài của hệ thống, từ đó đánh giá hành vi của bộ điều khiển nhúng trong điều kiện gần với thực tế. Một ví dụ tiêu biểu là nền tảng Typhoon HIL 404, được sử dụng để mô phỏng và kiểm thử các hệ thống điều khiển nhúng trong thời gian thực, giúp phát hiện sớm các lỗi thiết kế và thuật toán điều khiển trước khi triển khai phần cứng thật.

\begin{figure}[H]
    \centering
    \includegraphics[width=0.8\textwidth]{image/typhoon_HIL_404.png}
    \caption{Thiết bị Typhoon HIL 404 cho kiểm thử hệ thống nhúng}
    \label{fig:typhoon_HIL_404}
\end{figure}

Bên cạnh các thiết bị HIL chuyên dụng, nhiều nghiên cứu đã sử dụng các công cụ mô phỏng phần mềm như MATLAB/Simulink và LabVIEW để xây dựng môi trường HIL cho hệ thống nhúng. Các công cụ này cho phép mô hình hóa hành vi của cảm biến, ngoại vi và các khối chức năng, đồng thời kết nối trực tiếp với phần cứng nhúng thông qua các giao tiếp tiêu chuẩn. Cách tiếp cận này mang lại tính linh hoạt cao, chi phí thấp hơn và phù hợp với nghiên cứu, đào tạo cũng như phát triển các ứng dụng nhúng và IoT.

\begin{figure}[H]
    \centering
    \includegraphics[width=0.8\textwidth]{image/SimulinkRealTime.jpg}
    \caption{Matlab Simlink mô phỏng HIL của hệ thống thời gian thực}
    \label{fig:SimulinkRealTime}
\end{figure}

Tại Việt Nam, các hệ thống HIL chủ yếu được tiếp cận thông qua nghiên cứu và đào tạo, với trọng tâm là ứng dụng các công cụ mô phỏng phần mềm kết hợp với phần cứng nhúng phổ biến. Các giải pháp HIL chuyên dụng vẫn còn hạn chế do chi phí đầu tư cao, trong khi các mô hình HIL linh hoạt dựa trên phần mềm và vi điều khiển được đánh giá là phù hợp hơn với điều kiện nghiên cứu trong nước. Điều này cho thấy tiềm năng phát triển các hệ thống HIL cho thiết bị nhúng theo hướng đơn giản, dễ triển khai và chi phí thấp.

\section{Nhiệm vụ đề tài}
Từ thực trạng nghiên cứu trong và ngoài nước nêu trên, đề tài tập trung xây dựng một hệ thống mô phỏng phần cứng trong vòng lặp cho thiết bị nhúng, hướng đến mô phỏng các ngoại vi và giao tiếp phổ biến nhằm phục vụ kiểm thử firmware trong giai đoạn phát triển sớm.

Các nhiệm vụ cụ thể bao gồm:
\begin{itemize}
    \item Lập trình mô phỏng các ngoại vi trên hệ thống HIL: Triển khai các khối mô phỏng và giao tiếp cho các chuẩn phổ biến trong hệ thống nhúng, bao gồm SPI, I2C, CAN, ADC, DAC, GPIO và UART, nhằm tạo ra các tín hiệu đầu vào và xử lý phản hồi từ DUT trong quá trình kiểm thử.
    \item Đánh giá hoạt động của hệ thống HIL: Thực hiện các kịch bản kiểm thử để đánh giá hiệu năng, độ chính xác và tính ổn định của hệ thống HIL trong việc mô phỏng các ngoại vi và giao tiếp với DUT, từ đó xác định khả năng phát hiện lỗi và hỗ trợ phát triển firmware.
    \item Thiết kế và phát triển phần mềm giao diện hiển thị: Xây dựng giao diện người dùng trực quan để cấu hình, giám sát và điều khiển quá trình mô phỏng HIL, giúp người dùng dễ dàng thiết lập các kịch bản kiểm thử và theo dõi kết quả.
    \item Thiết kế và thi công phần cứng bo HIL: Phát triển bo mạch HIL với các thành phần phần cứng cần thiết để kết nối và giao tiếp với DUT, đảm bảo khả năng mở rộng và linh hoạt trong việc tích hợp các ngoại vi mô phỏng.
\end{itemize}

Thông qua việc hoàn thành các nhiệm vụ trên, đề tài kỳ vọng đóng góp một giải pháp mô phỏng phần cứng trong vòng lặp chi phí thấp, linh hoạt và hiệu quả cho việc kiểm thử thiết bị nhúng, hỗ trợ quá trình phát triển firmware trong các giai đoạn đầu.

